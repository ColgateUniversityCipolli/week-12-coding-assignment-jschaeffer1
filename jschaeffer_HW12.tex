\documentclass{article}\usepackage[]{graphicx}\usepackage[]{xcolor}
% maxwidth is the original width if it is less than linewidth
% otherwise use linewidth (to make sure the graphics do not exceed the margin)
\makeatletter
\def\maxwidth{ %
  \ifdim\Gin@nat@width>\linewidth
    \linewidth
  \else
    \Gin@nat@width
  \fi
}
\makeatother

\definecolor{fgcolor}{rgb}{0.345, 0.345, 0.345}
\newcommand{\hlnum}[1]{\textcolor[rgb]{0.686,0.059,0.569}{#1}}%
\newcommand{\hlsng}[1]{\textcolor[rgb]{0.192,0.494,0.8}{#1}}%
\newcommand{\hlcom}[1]{\textcolor[rgb]{0.678,0.584,0.686}{\textit{#1}}}%
\newcommand{\hlopt}[1]{\textcolor[rgb]{0,0,0}{#1}}%
\newcommand{\hldef}[1]{\textcolor[rgb]{0.345,0.345,0.345}{#1}}%
\newcommand{\hlkwa}[1]{\textcolor[rgb]{0.161,0.373,0.58}{\textbf{#1}}}%
\newcommand{\hlkwb}[1]{\textcolor[rgb]{0.69,0.353,0.396}{#1}}%
\newcommand{\hlkwc}[1]{\textcolor[rgb]{0.333,0.667,0.333}{#1}}%
\newcommand{\hlkwd}[1]{\textcolor[rgb]{0.737,0.353,0.396}{\textbf{#1}}}%
\let\hlipl\hlkwb

\usepackage{framed}
\makeatletter
\newenvironment{kframe}{%
 \def\at@end@of@kframe{}%
 \ifinner\ifhmode%
  \def\at@end@of@kframe{\end{minipage}}%
  \begin{minipage}{\columnwidth}%
 \fi\fi%
 \def\FrameCommand##1{\hskip\@totalleftmargin \hskip-\fboxsep
 \colorbox{shadecolor}{##1}\hskip-\fboxsep
     % There is no \\@totalrightmargin, so:
     \hskip-\linewidth \hskip-\@totalleftmargin \hskip\columnwidth}%
 \MakeFramed {\advance\hsize-\width
   \@totalleftmargin\z@ \linewidth\hsize
   \@setminipage}}%
 {\par\unskip\endMakeFramed%
 \at@end@of@kframe}
\makeatother

\definecolor{shadecolor}{rgb}{.97, .97, .97}
\definecolor{messagecolor}{rgb}{0, 0, 0}
\definecolor{warningcolor}{rgb}{1, 0, 1}
\definecolor{errorcolor}{rgb}{1, 0, 0}
\newenvironment{knitrout}{}{} % an empty environment to be redefined in TeX

\usepackage{alltt}
\usepackage[margin=1.0in]{geometry} % To set margins
\usepackage{amsmath}  % This allows me to use the align functionality.
                      % If you find yourself trying to replicate
                      % something you found online, ensure you're
                      % loading the necessary packages!
\usepackage{amsfonts} % Math font
\usepackage{fancyvrb}
\usepackage{hyperref} % For including hyperlinks
\usepackage[shortlabels]{enumitem}% For enumerated lists with labels specified
                                  % We had to run tlmgr_install("enumitem") in R
\usepackage{float}    % For telling R where to put a table/figure
\usepackage{natbib}        %For the bibliography
\bibliographystyle{apalike}%For the bibliography
\IfFileExists{upquote.sty}{\usepackage{upquote}}{}
\begin{document}


\begin{enumerate}
%%%%%%%%%%%%%%%%%%%%%%%%%%%%%%%%%%%%%%%%%%%%%%%%%%%%%%%%%%%%%%%%%%%%%%%%%%%%%%%%
%%%%%%%%%%%%%%%%%%%%%%%%%%%%%%%%%%%%%%%%%%%%%%%%%%%%%%%%%%%%%%%%%%%%%%%%%%%%%%%%
% Question 1
%%%%%%%%%%%%%%%%%%%%%%%%%%%%%%%%%%%%%%%%%%%%%%%%%%%%%%%%%%%%%%%%%%%%%%%%%%%%%%%%
%%%%%%%%%%%%%%%%%%%%%%%%%%%%%%%%%%%%%%%%%%%%%%%%%%%%%%%%%%%%%%%%%%%%%%%%%%%%%%%%
\item A group of researchers is running an experiment over the course of 30 months, 
with a single observation collected at the end of each month. Let $X_1, ..., X_{30}$
denote the observations for each month. From prior studies, the researchers know that
\[X_i \sim f_X(x),\]
but the mean $\mu_X$ is unknown, and they wish to conduct the following test
\begin{align*}
H_0&: \mu_X = 0\\
H_a&: \mu_X > 0.
\end{align*}
At month $k$, they have accumulated data $X_1, ..., X_k$ and they have the 
$t$-statistic
\[T_k = \frac{\bar{X} - 0}{S_k/\sqrt{n}}.\]
The initial plan was to test the hypotheses after all data was collected (at the 
end of month 30), at level $\alpha=0.05$. However, conducting the experiment is 
expensive, so the researchers want to ``peek" at the data at the end of month 20 
to see if they can stop it early. That is, the researchers propose to check 
whether $t_{20}$ provides statistically discernible support for the alternative. 
If it does, they will stop the experiment early and report support for the 
researcher's alternative hypothesis. If it does not, they will continue to month 
30 and test whether $t_{30}$ provides statistically discernible support for the
alternative.

\begin{enumerate}
  \item What values of $t_{20}$ provide statistically discernible support for the
  alternative hypothesis?
  
\begin{knitrout}\scriptsize
\definecolor{shadecolor}{rgb}{0.969, 0.969, 0.969}\color{fgcolor}\begin{kframe}
\begin{alltt}
 \hlkwd{library}\hldef{(tidyverse)}
\hlkwd{library}\hldef{(VGAM)}
\end{alltt}


{\ttfamily\noindent\color{warningcolor}{\#\# Warning: package 'VGAM' was built under R version 4.4.3}}

{\ttfamily\noindent\itshape\color{messagecolor}{\#\# Loading required package: stats4}}

{\ttfamily\noindent\itshape\color{messagecolor}{\#\# Loading required package: splines}}\begin{alltt}
\hlcom{#Question 1}

\hlcom{#Part a}
\hldef{t_val20} \hlkwb{=} \hlkwd{qt}\hldef{(}\hlnum{0.95}\hldef{,} \hlkwc{df}\hldef{=}\hlnum{19}\hldef{)}
\hldef{t_val20}
\end{alltt}
\begin{verbatim}
## [1] 1.729133
\end{verbatim}
\begin{alltt}
\hlcom{#Part b}
\hldef{t_val30} \hlkwb{=} \hlkwd{qt}\hldef{(}\hlnum{0.95}\hldef{,} \hlkwc{df}\hldef{=}\hlnum{29}\hldef{)}
\hldef{t_val30}
\end{alltt}
\begin{verbatim}
## [1] 1.699127
\end{verbatim}
\begin{alltt}
\hlcom{#Part c}
\hldef{R}\hlkwb{=}\hlnum{10000}
\hldef{err20} \hlkwb{=} \hlnum{0}
\hldef{err30} \hlkwb{=} \hlnum{0}

\hlkwa{for}\hldef{(i} \hlkwa{in} \hlnum{1}\hlopt{:}\hldef{R)\{}
\hlkwd{set.seed}\hldef{(}\hlnum{7272}\hlopt{+}\hldef{i)}
\hldef{x20} \hlkwb{=} \hlkwd{rlaplace}\hldef{(}\hlnum{20}\hldef{,}\hlnum{0}\hldef{,}\hlnum{4}\hldef{)} \hlcom{#Calculating sample}
\hldef{t20} \hlkwb{=} \hlkwd{mean}\hldef{(x20)} \hlopt{/} \hldef{(}\hlkwd{sd}\hldef{(x20)}\hlopt{/}\hlkwd{sqrt}\hldef{(}\hlnum{20}\hldef{))} \hlcom{#Calculating t20}
\hlkwa{if} \hldef{(t20}\hlopt{>}\hldef{t_val20)} \hlcom{#If type1 error, add 1}
  \hldef{err20}\hlkwb{=}\hldef{err20}\hlopt{+}\hlnum{1}

\hldef{x30} \hlkwb{=} \hlkwd{rlaplace}\hldef{(}\hlnum{30}\hldef{,}\hlnum{0}\hldef{,}\hlnum{4}\hldef{)} \hlcom{#Calculating sample}
\hldef{t30} \hlkwb{=} \hlkwd{mean}\hldef{(x30)} \hlopt{/} \hldef{(}\hlkwd{sd}\hldef{(x30)}\hlopt{/}\hlkwd{sqrt}\hldef{(}\hlnum{30}\hldef{))} \hlcom{#Calculating t30}
\hlkwa{if} \hldef{(t30}\hlopt{>}\hldef{t_val30)}
  \hldef{err30}\hlkwb{=}\hldef{err30}\hlopt{+}\hlnum{1}
\hldef{\}}
\hlcom{#Calculating percentage of t error}
\hldef{perc_err20} \hlkwb{=} \hldef{err20}\hlopt{/}\hldef{R}
\hldef{perc_err30}\hlkwb{=} \hldef{err30}\hlopt{/}\hldef{R}

\hldef{perc_err20}
\end{alltt}
\begin{verbatim}
## [1] 0.0507
\end{verbatim}
\begin{alltt}
\hldef{perc_err30}
\end{alltt}
\begin{verbatim}
## [1] 0.05
\end{verbatim}
\end{kframe}
\end{knitrout}

t values that satisfy $\left|{t}\right|>1.729$ provide statistically discernible support.

  \item What values of $t_{30}$ provide statistically discernible support for the
  alternative hypothesis?
  
  t values that satisfy $\left|{t}\right|>1.699$ provide statistically discernible support.
  
  \item Suppose $f_X(x)$ is a Laplace distribution with $a=0$ and $b=4.0$.
  Conduct a simulation study to assess the Type I error rate of this approach.\\
  \textbf{Note:} You can use the \texttt{rlaplace()} function from the \texttt{VGAM}
  package for \texttt{R} \citep{VGAM}.
  
  When simulating type I error, I calculating a 0.0507 rate of error when peeking at the data, and a 0.05 error rate when examining the data after 30 months. This suggests that looking at the data at 20 months increases the chance of a type I error.
  
  \item \textbf{Optional Challenge:} Can you find a value of $\alpha<0.05$ that yields a 
  Type I error rate of 0.05?
\end{enumerate}
%%%%%%%%%%%%%%%%%%%%%%%%%%%%%%%%%%%%%%%%%%%%%%%%%%%%%%%%%%%%%%%%%%%%%%%%%%%%%%%%
%%%%%%%%%%%%%%%%%%%%%%%%%%%%%%%%%%%%%%%%%%%%%%%%%%%%%%%%%%%%%%%%%%%%%%%%%%%%%%%%
% Question 2
%%%%%%%%%%%%%%%%%%%%%%%%%%%%%%%%%%%%%%%%%%%%%%%%%%%%%%%%%%%%%%%%%%%%%%%%%%%%%%%%
%%%%%%%%%%%%%%%%%%%%%%%%%%%%%%%%%%%%%%%%%%%%%%%%%%%%%%%%%%%%%%%%%%%%%%%%%%%%%%%%
  \item Perform a simulation study to assess the robustness of the $T$ test. 
  Specifically, generate samples of size $n=15$ from the Beta(10,2), Beta(2,10), 
  and Beta(10,10) distributions and conduct the following hypothesis tests against 
  the actual mean for each case (e.g., $\frac{10}{10+2}$, $\frac{2}{10+2}$, and 
  $\frac{10}{10+10}$).
  
\begin{knitrout}\scriptsize
\definecolor{shadecolor}{rgb}{0.969, 0.969, 0.969}\color{fgcolor}\begin{kframe}
\begin{alltt}
\hlcom{############################}
\hlcom{#####    QUESTION 2    #####}
\hlcom{############################}
\hldef{n}\hlkwb{=}\hlnum{15} \hlcom{#Initializing}
\hldef{R}\hlkwb{=}\hlnum{10000}
\hldef{mean1} \hlkwb{=} \hlnum{10}\hlopt{/}\hlnum{12}
\hldef{mean2} \hlkwb{=} \hlnum{2}\hlopt{/}\hlnum{12}
\hldef{mean3} \hlkwb{=} \hlnum{10}\hlopt{/}\hlnum{20}
\hldef{err1_left}\hlkwb{=}\hlnum{0}
\hldef{err2_left}\hlkwb{=}\hlnum{0}
\hldef{err3_left}\hlkwb{=}\hlnum{0}

\hldef{err1_right}\hlkwb{=}\hlnum{0}
\hldef{err2_right}\hlkwb{=}\hlnum{0}
\hldef{err3_right}\hlkwb{=}\hlnum{0}

\hldef{err1_two}\hlkwb{=}\hlnum{0}
\hldef{err2_two}\hlkwb{=}\hlnum{0}
\hldef{err3_two}\hlkwb{=}\hlnum{0}
\hlkwa{for}\hldef{(i} \hlkwa{in} \hlnum{1}\hlopt{:}\hldef{R)\{}
  \hlkwd{set.seed}\hldef{(}\hlnum{7272}\hlopt{+}\hldef{i)}
  \hldef{sample1} \hlkwb{=} \hlkwd{rbeta}\hldef{(n,} \hlnum{10}\hldef{,} \hlnum{2}\hldef{)} \hlcom{#Calculating samples}
  \hldef{sample2} \hlkwb{=} \hlkwd{rbeta}\hldef{(n,} \hlnum{2}\hldef{,} \hlnum{10}\hldef{)}
  \hldef{sample3} \hlkwb{=} \hlkwd{rbeta}\hldef{(n,} \hlnum{10}\hldef{,} \hlnum{10}\hldef{)}

  \hlcom{#Calculating t values}
  \hldef{t1} \hlkwb{=} \hldef{(}\hlkwd{mean}\hldef{(sample1)}\hlopt{-}\hldef{mean1)}\hlopt{/}\hldef{(}\hlkwd{sd}\hldef{(sample1)}\hlopt{/}\hlkwd{sqrt}\hldef{(n))}
  \hldef{t2} \hlkwb{=} \hldef{(}\hlkwd{mean}\hldef{(sample2)}\hlopt{-}\hldef{mean2)}\hlopt{/}\hldef{(}\hlkwd{sd}\hldef{(sample2)}\hlopt{/}\hlkwd{sqrt}\hldef{(n))}
  \hldef{t3} \hlkwb{=} \hldef{(}\hlkwd{mean}\hldef{(sample3)}\hlopt{-}\hldef{mean3)}\hlopt{/}\hldef{(}\hlkwd{sd}\hldef{(sample3)}\hlopt{/}\hlkwd{sqrt}\hldef{(n))}
  \hlcom{#Checking if type 1 error occurred for left test}
  \hlkwa{if} \hldef{(t1}\hlopt{<}\hlkwd{qt}\hldef{(}\hlnum{0.05}\hldef{,} \hlkwc{df}\hldef{=}\hlnum{14}\hldef{))}
    \hldef{err1_left}\hlkwb{=}\hldef{err1_left}\hlopt{+}\hlnum{1}
  \hlkwa{if} \hldef{(t2}\hlopt{<}\hlkwd{qt}\hldef{(}\hlnum{0.05}\hldef{,} \hlkwc{df}\hldef{=}\hlnum{14}\hldef{))}
    \hldef{err2_left}\hlkwb{=}\hldef{err2_left}\hlopt{+}\hlnum{1}
  \hlkwa{if} \hldef{(t3}\hlopt{<}\hlkwd{qt}\hldef{(}\hlnum{0.05}\hldef{,} \hlkwc{df}\hldef{=}\hlnum{14}\hldef{))}
    \hldef{err3_left}\hlkwb{=}\hldef{err3_left}\hlopt{+}\hlnum{1}


  \hlcom{#Checking if type 1 error occurred for right test}
  \hlkwa{if} \hldef{(t1}\hlopt{>}\hlkwd{qt}\hldef{(}\hlnum{0.95}\hldef{,} \hlkwc{df}\hldef{=}\hlnum{14}\hldef{))}
    \hldef{err1_right}\hlkwb{=}\hldef{err1_right}\hlopt{+}\hlnum{1}
  \hlkwa{if} \hldef{(t2}\hlopt{>}\hlkwd{qt}\hldef{(}\hlnum{0.95}\hldef{,} \hlkwc{df}\hldef{=}\hlnum{14}\hldef{))}
    \hldef{err2_right}\hlkwb{=}\hldef{err2_right}\hlopt{+}\hlnum{1}
  \hlkwa{if} \hldef{(t3}\hlopt{>}\hlkwd{qt}\hldef{(}\hlnum{0.95}\hldef{,} \hlkwc{df}\hldef{=}\hlnum{14}\hldef{))}
    \hldef{err3_right}\hlkwb{=}\hldef{err3_right}\hlopt{+}\hlnum{1}


  \hlcom{#Checking if type 1 error occurred for two tailed test}
  \hlkwa{if} \hldef{(}\hlkwd{abs}\hldef{(t1)}\hlopt{>}\hlkwd{qt}\hldef{(}\hlnum{0.975}\hldef{,} \hlkwc{df}\hldef{=}\hlnum{14}\hldef{))}
    \hldef{err1_two}\hlkwb{=}\hldef{err1_two}\hlopt{+}\hlnum{1}
  \hlkwa{if} \hldef{(}\hlkwd{abs}\hldef{(t2)}\hlopt{>}\hlkwd{qt}\hldef{(}\hlnum{0.975}\hldef{,} \hlkwc{df}\hldef{=}\hlnum{14}\hldef{))}
    \hldef{err2_two}\hlkwb{=}\hldef{err2_two}\hlopt{+}\hlnum{1}
  \hlkwa{if} \hldef{(}\hlkwd{abs}\hldef{(t3)}\hlopt{>}\hlkwd{qt}\hldef{(}\hlnum{0.975}\hldef{,} \hlkwc{df}\hldef{=}\hlnum{14}\hldef{))}
    \hldef{err3_two}\hlkwb{=}\hldef{err3_two}\hlopt{+}\hlnum{1}

\hldef{\}}

\hldef{err1_left}\hlopt{/}\hldef{R}
\end{alltt}
\begin{verbatim}
## [1] 0.0303
\end{verbatim}
\begin{alltt}
\hldef{err2_left}\hlopt{/}\hldef{R}
\end{alltt}
\begin{verbatim}
## [1] 0.0769
\end{verbatim}
\begin{alltt}
\hldef{err3_left}\hlopt{/}\hldef{R}
\end{alltt}
\begin{verbatim}
## [1] 0.0477
\end{verbatim}
\end{kframe}
\end{knitrout}

  \begin{enumerate}
    \item What proportion of the time do we make an error of Type I for a
    left-tailed test?
 
 For the left-tailed test, we make a type I error at a rate of 0.0303, 0.0769, and 0.0477 for respective parameter values of Beta(10,2), Beta(2,10), and Beta(10,10).

    \item What proportion of the time do we make an error of Type I for a
    right-tailed test?
\begin{knitrout}\scriptsize
\definecolor{shadecolor}{rgb}{0.969, 0.969, 0.969}\color{fgcolor}\begin{kframe}
\begin{alltt}
\hldef{err1_right}\hlopt{/}\hldef{R}
\end{alltt}
\begin{verbatim}
## [1] 0.0796
\end{verbatim}
\begin{alltt}
\hldef{err2_right}\hlopt{/}\hldef{R}
\end{alltt}
\begin{verbatim}
## [1] 0.029
\end{verbatim}
\begin{alltt}
\hldef{err3_right}\hlopt{/}\hldef{R}
\end{alltt}
\begin{verbatim}
## [1] 0.049
\end{verbatim}
\end{kframe}
\end{knitrout}
     For the right-tailed test, we make a type I error at a rate of 0.0796, 0.0290, and 0.0490 for respective parameter values of Beta(10,2), Beta(2,10), and Beta(10,10).
    \item What proportion of the time do we make an error of Type I for a
    two-tailed test?
    
\begin{knitrout}\scriptsize
\definecolor{shadecolor}{rgb}{0.969, 0.969, 0.969}\color{fgcolor}\begin{kframe}
\begin{alltt}
\hldef{err1_two}\hlopt{/}\hldef{R}
\end{alltt}
\begin{verbatim}
## [1] 0.0619
\end{verbatim}
\begin{alltt}
\hldef{err2_two}\hlopt{/}\hldef{R}
\end{alltt}
\begin{verbatim}
## [1] 0.0582
\end{verbatim}
\begin{alltt}
\hldef{err3_two}\hlopt{/}\hldef{R}
\end{alltt}
\begin{verbatim}
## [1] 0.0503
\end{verbatim}
\end{kframe}
\end{knitrout}
     For the two-tailed test, we make a type I error at a rate of 0.0619, 0.0582, and 0.0503 for respective parameter values of Beta(10,2), Beta(2,10), and Beta(10,10).
     
    \item How does skewness of the underlying population distribution effect
    Type I error across the test types?
    
    We can see that skewness can cause left-tailed versus right-tailed to have a noticeably different type I error rates. For the left skewed beta(10,2) distribution, there are more type I errors when conducting a right-tailed test compared to a left-tailed test. The opposite is true for the right skewed beta(2,10) distribution. For the beta(10,10) distribution, where the distribution is not skewed, there is not a large difference between type I error across differing test types. Further, we can notice that the type I error rate for the differently skewed distributions is similar for the two-tailed test because their increased likelihoods for opposing tests average out.
    
  \end{enumerate}
%%%%%%%%%%%%%%%%%%%%%%%%%%%%%%%%%%%%%%%%%%%%%%%%%%%%%%%%%%%%%%%%%%%%%%%%%%%%%%%%
%%%%%%%%%%%%%%%%%%%%%%%%%%%%%%%%%%%%%%%%%%%%%%%%%%%%%%%%%%%%%%%%%%%%%%%%%%%%%%%%
% End Document
%%%%%%%%%%%%%%%%%%%%%%%%%%%%%%%%%%%%%%%%%%%%%%%%%%%%%%%%%%%%%%%%%%%%%%%%%%%%%%%%
%%%%%%%%%%%%%%%%%%%%%%%%%%%%%%%%%%%%%%%%%%%%%%%%%%%%%%%%%%%%%%%%%%%%%%%%%%%%%%%%
\end{enumerate}
\bibliography{bibliography}
\end{document}
